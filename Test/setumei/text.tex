\documentclass[autodetect-engine,dvipdfmx-if-dvi,ja=standard]{bxjsarticle}

% 二段組にするとき
% \documentclass[twocolumn,autodetect-engine,dvipdfmx-if-dvi,ja=standard]{bxjsarticle}

\usepackage{graphicx}        %図を表示するのに必要
\usepackage{color}           %jpgなどを表示するのに必要
\usepackage{amsmath,amssymb} %数学記号を出すのに必要
\usepackage{setspace}
% \usepackage{eclclass}
\usepackage{cases}
\usepackage{here}
\usepackage{fancyhdr}
\usepackage{ascmac}
\usepackage{lscape}
\usepackage{titlesec}

% 文書全体のスタイルを設定(主に余白)
\setlength{\topmargin}{-0.3in}
\setlength{\oddsidemargin}{0pt}
\setlength{\evensidemargin}{0pt}
\setlength{\textheight}{44\baselineskip}

\titleformat*{\subsection}{\normalsize\bfseries}

% 行頭の字下げをしない
\parindent = 0pt

% ヘッダとフッタの設定
\lhead{電気通信}
\chead{中間試験過去問題集}
\rhead{}
\lfoot{}
\cfoot{-\thepage-} % ページ数
\rfoot{}

% 式の番号を(senction_num.num)のようにする
\makeatletter
\@addtoreset{equation}{section}
\def\theequation{\thesection.\arabic{equation}}
\makeatother

% 画像の貼り付けを簡単にする
\newcommand{\pic}[2]
{
  \begin{figure}[H]
    \begin{center}
      \includegraphics[scale=#2]{#1}
    \end{center}
  \end{figure}
}

% 単位の記述を簡単にする
\newcommand{\unit}[1]
{
  \, [\mathrm{#1}]
}
\begin{document}
% \maketitle
\pagestyle{fancy}
\section{説明問題}
\subsection{呼損率の取りうる値の範囲とその状態を説明せよ.(H30)}
 「呼」が発生した場合に交換器によって全く処理されず,通信できない状態の場合は$B=1$となり,逆にすぐ繋がる最良の場合は$B=0$となる.値の取りうる範囲は$0\leqq B \leqq1$\\

\subsection{呼損率の定義を説明せよ.また,呼損率が1や0の場合の状態について説明せよ.(課題,H27,H25,H23)}
 呼損率とは「呼」が発生した時に回線が繋がらない確率のこと.\\
 呼損率が「1」の場合は回線がまったく繋がない.\\
 呼損率が「0」の場合は必ず回線が繋がる.\\

\subsection{AM変調とFM変調の特徴を比較して説明せよ.(H30)}
 AM変調は振幅により変調を行うため,回路を容易に作れる.また,FM変調よりも電波の伝送距離が長い.FM変調は周波数変調であるため,ノイズの影響を受けにくい.

\subsection{FM変調において必要とされる帯域幅はいくらか,根拠となる理由も含め説明せよ.(課題,H30,H28)}
 FM変調は帯域幅において$99.9\%$の電力を$\left( f_c + f_p \right) \left( f_c + f_p \right)$間に含むため,必要となる帯域幅は$2 f_p$となる.\\

\subsection{アームストロング式のPM変調装置やFM変調装置の特徴を説明せよ.(H27,H25)}
 アームストロング式では,AM変調装置に位相シフトを加えて搬送波を$\frac{\pi}{2}$ずらすことでPM変調装置となり,積分装置を加えて信号を積分することでFM変調装置となる.このようにAM変調装置を流用できるという利点がある.\\

\subsection{AM変調とFM変調のSN比について,電波強度の強い場合と弱い場合に分け,リミッタ効果の影響も含めて説明せよ(H25)}
 FM変調波の電圧(振幅)にはデータが含まれていないため,リミッタにより電圧の制限をしても良い.電波強度の強い場合には電圧の制限をすることでノイズの影響を少なくでき,AM波と比べてSN比はよくなる.\\
 電波強度が弱い場合には,制限してもノイズの影響は少なくなりにくく,AM波よりSN比がよくなるとは言い切れない.\\

\subsection{ダイオードを使用したAM復調回路の動作と,ダイアゴナルクリッピング(けさ切りひずみ)について説明せよ.(H28,H27,H25,H23)}
\vspace{4cm}
 AM復調回路は図1となり,その波形は図2のようになる.\\
 Cの電荷のの放電が遅くなるとき,図2の矢印の部分の様に出力波形が包絡線から外れてひずみが生じる.この現象のことをダイアゴナルクリッピングという.\\

\subsection{次のようなFMスロープ検波器の特徴を復調特性について説明せよ.(課題,H28,H27,H25)}
\vspace{4cm}
 この検波器の出力特性は図のようなS字特性となる.\\
 2つの共振回路を組み合わせることで非線形な領域を少なくし,直線的なスロープを作りひずみを低減している\\.
 これの入力は電圧(振幅)を制限する必要があり,一定に保つ必要がある.\\

\subsection{一般に周期的時間関数はフーリエ級数,過渡的時間はフーリエ変換で解析を行うがホワイトノイズのようなランダムな関数を解析する場合,どのように扱えば良いかを述べよ.(H25)}
\vspace{4cm}
 図の様に,ある周期だけを取り出し,それを連続して並べることで周期的時間関数として扱うことができ,フーリエ級数展開ができる.\\
 また,取り出したものを単体で見ると,過渡的時間関数として扱え,フーリエ変換ができる.\\

\subsection{ホワイトノイズを含む電気信号をフーリエ級数やフーリエ変換を用いて解析したい場合,この連続性ノイズをどのように扱えば良いか.(H20)}
 フーリエ級数で解析する場合は,ノイズが十分に長い手記で同じ波形を繰り返しているとみなして扱う.\\
 フーリエ変換で解析する場合は,解析する範囲にだけ信号が存在するとみなして扱う.\\

\subsection{ホワイトノイズの特徴とその平均電力について説明せよ.(H20)}

\subsection{ノイズ指数が等しく増幅度の異なる回路を2個用意して回路を構成する場合,接続はどのようにすれば良いか説明せよ.(H20)}

\subsection{AD変換における標本化と量子化について「離散化」の観点から説明せよ.(課題,H28)}

\subsection{PAM変調,PWM変調,PFM変調について説明し,具体的な波形をかけ(H20,H19)}

\newpage
\section{計算問題}
\subsection{電話240台が接続されている交換機において,電話機1台当たりの呼量が1時間について6分であった.この電話回路における呼損率を0.7とした場合,この交換機の出線は何本必要か.ただし,1本の出線が処理可能な呼量は0.8[アーラン]とする.(H27,H25,H23)}
\vspace{7cm}

\subsection{1時間に平均10分使用する電話機120台が電話交換機に接続され,呼損率は0.4であった.ここで,さらに電話機が60台追加された場合の呼損率を求めよ.(H30)}
\vspace{7cm}

\subsection{1台当たり呼量0.15[アーラン]の電話器180台が接続されている電話交換機があり,その出線数は12本であった.出線1本当たりの処理可能な呼量を0.8[アーラン]とした場合の呼損率を計算せよ.(H28)}
\vspace{7cm}

\newpage
\subsection{130台の電話が交換機に接続され1台当たりの通話時間は1時間当たり4分である.交換機の出線が3本で,1台当たりの処理呼量が0.7[E]の場合の電話回線の呼損率を求めよ.(課題)}
\vspace{7cm}

\subsection{周波数帯域が150[kHz]の通信回線において,通信容量900[kbps]を得るためには回線のS/N比はいくら必要か(H23)}
\vspace{7cm}

\subsection{SN比が31倍の回線において,通信容量2[Mbps]を得るために必要な周波数帯域を求めよ.(H30)}
\vspace{7cm}

\newpage
\subsection{周波数帯域が700[kHz]の通信回線において,通信容量2.5[Mbps]を得るためには回線のSN比はいくら必要か求めよ.(H28)}
\vspace{7cm}

\subsection{周波数帯域が2.5[MHz]の通信回線におけるSN比が15倍であった.この通信容量C[bps]を求めよ.(課題,H27,H25)}
\vspace{7cm}

\subsection{搬送波が$c(t) = \cos(\omega_c t)$,信号波が$s(t) = A \cos(pt)$であるAM変調波を表す式を計算し,側波帯の存在を式で表せ.(H28,H25,H23)}
\vspace{7cm}

\newpage
\subsection{搬送波が$c(t) = 10\cos(\omega_c t)$,信号波が$s(t) = 2 \cos(pt)$であるAM変調波を表す式を計算し,側波帯の存在を式で表せ.(H27)}
\vspace{7cm}

\subsection{搬送波が$c(t) = C\cos(\omega_c t)$,信号波が$s(t) = \cos(pt)$とするアナログ変調において,AM波,FM波を表す式を書け(課題,H30)}
\vspace{7cm}

\subsection{信号波が$s(t) = \cos(pt)$,搬送波が$c(t) = C \cos(\omega_c t)$を用いてPM変調を行なった.このPM変調波PM(t)を表す式を示せ.ただし最大位相偏移は$\Theta$とする.(H27)}
\vspace{7cm}

\newpage
\subsection{ラジオ放送において,最大周波数が$f_p[Hz]$の信号を変調する場合,必要とされる帯域幅はいくらになるか.AM変調とFM変調を比較しその根拠を述べよ.(H27)}
\vspace{7cm}

\subsection{AM変調における片側の側波帯と搬送波の電力比を求めよ.ただし,変調度(変調指数)はm=0.3とする.(H27,H25,H23)}
\vspace{7cm}

\newpage
\subsection{図の波形g(t)をフーリエ級数展開し標本化関数が含まれることを示せ,またそのスペクトラムを図示せよ.(課題,H30,H28,H19)}
\vspace{11cm}

\subsection{図の非周期パルス波g(t)のフーリエ変換G(f)を求めて標本化関数$\sin(x) / x$が含まれてることを示し,そのスペクトルの概形を描け.(課題,H23,H20)}
\vspace{11cm}

\newpage
\subsection{$\cos \theta$と$\sin \theta$の相互相関関数を計算で求めよ.(H19)}
\vspace{11cm}

\subsection{$\sin \theta$と$\cos(\theta + \tau)$の二つの関数について,関数の積を$0$から$2 \pi$まで積分し,相互相関関数を求めよ.(H20)}
\vspace{11cm}

\newpage
\subsection{純抵抗Rから生ずるノイズ電圧の実効値を表す式を示せ(課題)}
\vspace{7cm}

\subsection{図のような整合された2段従続回路の全体のノイズ指数$F_t$を求めよ.またその結果からどのようなことが解るか説明を加えよ.(H30,H28,H23,H20,H19)}
\vspace{7cm}

\subsection{教科書P43考察的問題2.4の解の,(2.51)から(2.56)に至る記述において,間違いを指摘せよ.(課題)}
\vspace{7cm}

\newpage
\subsection{$300Hz$から$3.4kHz$の帯域で最大振幅$3V_{p-p}$のアナログ信号を9bitでAD変換したい.必要なサンプリング周波数と量子化誤差の最大電圧はいくらになるか.(H19)}
\vspace{7cm}

\subsection{$1kHz$〜$15kHz$,最大振幅$5V_{p-p}$のアナログ信号を10bitでAD変換したい.量子化誤差と標本周波数を求め考察せよ.(H20)}
\vspace{7cm}

\subsection{30Hz〜15kHzの帯域の音楽信号(最大振幅$5V_{p-p}$)をAD変換したい.必要な標本化周波数を答えよ.また,8bitでAD変換した場合の量子化誤差を求めよ.(H30)}

\end{document}
