\input{./setting.tex}
\begin{document}
% \maketitle
\pagestyle{fancy}
\section{説明問題}
\subsection{分布定数回路において,電信方程式と波動方程式の違いを説明せよ.(H27,H26)}
 電信方程式:モールス信号のようにON/OFFのみ全ての信号に対応する.\\.
 波動方程式:正弦波の連続信号にのみ対応.

\subsection{光ファイバーケーブルについてシングルモードとマルチモードの違いを説明せよ.(H30,H28, H23)}
 マルチモードはケーブル内での反射により複数の光路が存在し,それぞれによって遅延が発生するため使用できる帯域は比較的狭い.\\
 シングルモードは構造上,ケーブル内での反射がないため,マルチモードのような制約がなく,広帯域で長距離の伝送に適している.

\subsection{放射電磁界における偏波とは何かを説明せよ.(H23)}

\subsection{地上から垂直に高さ80mのAM波用ユニポールアンテナを建てた.電波の偏波とは何かを説明し,このアンテナの偏波方向を答えよ.(H30)}
 電波の偏波とは電界の振動である.このアンテナは地面に対して垂直であることからその偏波方向も垂直である.

\subsection{図のアンテナの名称と特徴を説明せよ.(H20)}

\subsection{図のアンテナの名称と特徴を説明せよ.(H20)}

\subsection{図のような中波AM放送用アンテナの特徴を説明せよ.また,放送用搬送波周波数が1200kHzの場合,アンテナ高さはどの程度必要か求めよ.(H23)}

\subsection{中波放送(いわゆるAM放送)は電離層の影響を受け,昼と夜とでは電波伝搬の様子が異なる,このことについて具体的に説明せよ.(H20)}

\subsection{短波放送は中波放送よりも遠方に届くという特徴がある.この理由を説明せよ.(H20)}

\subsection{中波放送(AM放送)の電波伝搬に関わる電離層の影響について説明せよ.(H30,H28)}
 電離層は地上からの位置が低い層から順にD層,E層,F層に分けられる.中波放送に用いられる電波はE層で反射し,D層で吸収される.しかし,夜間はD層が存在しないため,昼間よりも遠方と通信できる.

\subsection{MF帯,VHF帯,UHF帯の周波数範囲と実際に使用されている通信方式とメディアの具体例を記述せよ.(H23)}

\subsection{EMCについて説明せよ.(H30,H23)}
電磁気学的両立性のこと.動植物や人間の生活環境を含め,全ての社会構成要素が正常に動作することを考える学術領域である.通信におけるEMCには,ノイズを抑制または発生しないようにすることなどが含まれる.

\subsection{BERについて説明せよ.(H30)}
 信号中のビットが正しく通信されない確率.SN比に伴って大きくなる.

\subsection{ノイズイミュニティとは何か説明せよ.(H23)}

\subsection{誤り検出符号と前方向誤り訂正符号について比較して長所短所を説明せよ.(H30,
H28)}
 誤り検出符号は受信側が誤りを検出した場合に送信側へ再送要求し,正しいデータを得る.\\
 対して,前方向誤り検出符号は受信側が誤りを訂正する.\\
 前者は冗長ビットが1ビットですむが,再送処理には距離に応じた時間を要する.一方,後者は再送要求を行わないため,高速で通信できるが,冗長ビットは3ビット以上必要である.

\subsection{ハミング距離について説明せよ.(H23)}

\subsection{デジタルデータ変調方式のASK,FSK,PSKの各方式について等しい符号誤り率の条件下でSN比の関係を比較し説明せよ.(H28)}

\subsection{デジタルデータを変調する場合,符号誤り率とSN比の関係からPSK方式がASKやFSKと比較して有利とされる根拠を説明せよ.(H23,H21)}

\subsection{下の誤り訂正符号を用いてデータ通信を行った.受信データに誤りがあれば訂正せよ.(H23)}

\subsection{スペクトラム拡散方式の特徴を2項目以上述べよ.(H30,H28)}
 広帯域に渡って低いSN比で通信するため,低電力である.\\
 ノイズに近い暗号化方式を用いて通信するため,セキュリティ面でも優れている.

\subsection{スペクトラム拡散方式の通信では擬似ランダム符号であるM系列が使用されるが,この符号を用いることによりどのような特徴が得られるか説明せよ.(H23)}

\newpage
\section{計算問題}
\subsection{図の$R$[$\mathrm{\Omega}$/m],$L$[H/m],$G$[S/m],$C$[F/m]の定数を持つ分布定数回路において,以下の電信方程式これをもとに正弦波に限定した波動方程式を求めよ.(H25,H24,H23)}
\vspace{7cm}

\subsection{分布定数回路における波動方程式の電圧解は次式で与えられる.この式を用い無限長線路での進行波速度である位相速度を求めよ.(H30,H28,H27,H26,H25,H24)}
\vspace{7cm}

\subsection{比誘電率 2 の絶縁物を使用した,内部導体の直径0.7mm,外部導体の内径3mmの同軸ケーブルがある.この同軸ケーブルの特性インピーダンスを求めよ.(H30,H28,H23)}
\vspace{7cm}

\newpage
\subsection{特性インピーダンス65$\mathrm{\Omega}$の平行線路に特性インピーダンス35$\mathrm{\Omega}$の線路を接続した.この時,接点における電圧反射係数$r$の値と35$\mathrm{\Omega}$側の電圧定在波比を求めよ.(H30)}
\vspace{7cm}

\subsection{TE$_10$モードの内径寸法比2:1,遮断周波数10GHzの方形導波管の内側寸法を求めよ.(H23)}
\vspace{7cm}

\subsection{TE$_10$モードの内径寸法が12.5mm*25mmの方形導波管の遮断周波数を求めよ.また,この導波管に10GHzの電磁波を通した場合の館内波長を求めよ.(H30,H28)}
\vspace{7cm}

\newpage
\subsection{地上から垂直に高さ80mのAM波用ユニポールアンテナを建てた.(H30)}
\subsubsection{このアンテナでもっとも効率よく電波を放射できる搬送周波数を求めよ.(H30)}
\vspace{7cm}
\subsubsection{このアンテナに高周波電流2.7Aを供給した時,水平方向の距離90kmの地点での電界強度と磁界強度を求めよ.(H30)}
\vspace{7cm}

\subsection{半波長ダイポールアンテナにおいて,1.5Aの高周波電流を給電した.正面水平方向の80km離れた地点の電界強度と磁界強度を求めよ.ただし,空間の電波伝搬固有インピーダンスを120$\pi\mathrm{\Omega}$とする.(H28)}
\vspace{7cm}

\newpage
\subsection{「1011 0111 0011」 と 「0011 0101 1011」 の符号間距離を求めよ (H30)}
\vspace{7cm}

\subsection{スペクトラム拡散方式を用いて受信側のSN比4倍,周波数帯域3MHzで通信した時の情報通信容量を求めよ.(H30,H28)}
\vspace{7cm}

\subsection{24kHzの搬送波1波長について位相変化12通り,振幅変動3通りの多値変調を行った.この場合のデータの情報容量を求めよ.(H23)}
\vspace{7cm}

\end{document}
