\documentclass[autodetect-engine,dvipdfmx-if-dvi,ja=standard]{bxjsarticle}

% 二段組にするとき
% \documentclass[twocolumn,autodetect-engine,dvipdfmx-if-dvi,ja=standard]{bxjsarticle}

\usepackage{graphicx}        %図を表示するのに必要
\usepackage{color}           %jpgなどを表示するのに必要
\usepackage{amsmath,amssymb} %数学記号を出すのに必要
\usepackage{setspace}
% \usepackage{eclclass}
\usepackage{cases}
\usepackage{here}
\usepackage{fancyhdr}
\usepackage{ascmac}
\usepackage{lscape}
\usepackage{titlesec}

% 文書全体のスタイルを設定(主に余白)
\setlength{\topmargin}{-0.3in}
\setlength{\oddsidemargin}{0pt}
\setlength{\evensidemargin}{0pt}
\setlength{\textheight}{44\baselineskip}

\titleformat*{\subsection}{\normalsize\bfseries}

% 行頭の字下げをしない
\parindent = 0pt

% ヘッダとフッタの設定
\lhead{電気通信}
\chead{中間試験過去問題集}
\rhead{}
\lfoot{}
\cfoot{-\thepage-} % ページ数
\rfoot{}

% 式の番号を(senction_num.num)のようにする
\makeatletter
\@addtoreset{equation}{section}
\def\theequation{\thesection.\arabic{equation}}
\makeatother

% 画像の貼り付けを簡単にする
\newcommand{\pic}[2]
{
  \begin{figure}[H]
    \begin{center}
      \includegraphics[scale=#2]{#1}
    \end{center}
  \end{figure}
}

% 単位の記述を簡単にする
\newcommand{\unit}[1]
{
  \, [\mathrm{#1}]
}
\begin{document}
% \maketitle
\pagestyle{fancy}
\section{説明問題}
\subsection{分布定数回路において,電信方程式と波動方程式の違いを説明せよ.(H27,H26)}
 電信方程式:モールス信号のようにON/OFFのみ全ての信号に対応する.\\
 波動方程式:正弦波の連続信号にのみ対応.\\

\subsection{光ファイバーケーブルについてシングルモードとマルチモードの違いを説明せよ.(H30,H28, H23)}
 マルチモードはケーブル内での反射により複数の光路が存在し,それぞれによって遅延が発生するため使用できる帯域は比較的狭い.\\
 シングルモードは構造上,ケーブル内での反射がないため,マルチモードのような制約がなく,広帯域で長距離の伝送に適している.\\

\subsection{放射電磁界における偏波とは何かを説明せよ.(H23)}
 偏波とは電界の振動である.電界の振動面が大地に対して水平なものを水平偏波,垂直なものを垂直偏波と呼ぶ.また,水平偏波と垂直偏波の電気的位相差が90degでかつ,振幅が等しい偏波を円偏波という.\\

\subsection{地上から垂直に高さ80mのAM波用ユニポールアンテナを建てた.電波の偏波とは何かを説明し,このアンテナの偏波方向を答えよ.(H30)}
 電波の偏波とは電界の振動である.このアンテナは地面に対して垂直であることからその偏波方向も垂直である.\\

\subsection{図のアンテナの名称と特徴を説明せよ.(H20)}
 名称:ヘリカルアンテナ\\
 波長がアンテナよりも長いときは垂直偏波となる.\\
らせん状に巻いてあるため,アンテナの実質的な長さ\\
を抑えられる.\\
 波長がアンテナよりも短くなると円偏波となる.\\

\subsection{図のアンテナの名称と特徴を説明せよ.(H20)}
 名称:トップローディングアンテナ\\
 長波,中波放送用のアンテナで垂直偏波.地面に接地\\
を行うことでアンテナの全長が$\lambda / 2$必要なところを$\lambda / 4$まで\\
短くできる.また,アンテナの先端にトップローディング\\
をつけることでアンテナの全長をより抑えることができる.\\

\subsection{中波放送(いわゆるAM放送)は電離層の影響を受け,昼と夜とでは電波伝搬の様子が異なる,このことについて具体的に説明せよ.(H20)}
 昼間は電離層のD層により中波放送の電波が吸収される.よって,遠方の中波放送は地表を回折してくる電波のみ受信できる.\\
 夜間はD層が消え,E層により中波放送の電波は反射される.よって,昼間は受信できないような遠方の電波も受信できることがある.\\

\subsection{短波放送は中波放送よりも遠方に届くという特徴がある.この理由を説明せよ.(H20)}
 短波は中波と比較して直進性が強く,常にD層を通り抜け主にF層で反射される.このため,中波放送に比べ長距離伝送を実現できる.\\

\subsection{中波放送(AM放送)の電波伝搬に関わる電離層の影響について説明せよ.(H30,H28)}
 電離層は地上からの位置が低い層から順にD層,E層,F層に分けられる.中波放送に用いられる電波はE層で反射し,D層で吸収される.しかし,夜間はD層が存在しないため,昼間よりも遠方と通信できる.\\

\subsection{MF帯,VHF帯,UHF帯の周波数範囲と実際に使用されている通信方式とメディアの具体例を記述せよ.(H23)}
 MF 300kHz - 3MHz,AMラジオ\\
 VHF 30MHz - 300MHz,アナログテレビや放送FMラジオ\\
 UHF 300MHz - 3GHz,デジタルテレビや放送携帯電話\\

\subsection{EMCについて説明せよ.(H30,H23)}
電磁気学的両立性のこと.動植物や人間の生活環境を含め,全ての社会構成要素が正常に動作することを考える学術領域である.通信におけるEMCには,ノイズを抑制または発生しないようにすることなどが含まれる.\\

\subsection{BERについて説明せよ.(H30)}
 信号中のビットが正しく通信されない確率.SN比に伴って大きくなる.\\

\subsection{ノイズイミュニティとは何か説明せよ.(H23)}
 ノイズに対する免疫性のこと.一般には耐ノイズ性と呼ばれる.通信を行う上でノイズの影響は不可避であるため,耐ノイズ性を高めることは課題である.

\subsection{誤り検出符号と前方向誤り訂正符号について比較して長所短所を説明せよ.(H30,
H28)}
 誤り検出符号は受信側が誤りを検出した場合に送信側へ再送要求し,正しいデータを得る.\\
 対して,前方向誤り検出符号は受信側が誤りを訂正する.\\
 前者は冗長ビットが1ビットですむが,再送処理には距離に応じた時間を要する.一方,後者は再送要求を行わないため,高速で通信できるが,冗長ビットは3ビット以上必要である.\\

\subsection{ハミング距離について説明せよ.(H23)}
 符号間距離ともいわれ,符号$x$と$y$を仮定するとき,$d=\sum_{i=1}^{n}(x_i+y_i)$であらわされる.2つの符号の各桁を比較したときの異なるビットの数を表している.\\

\subsection{デジタルデータを変調する場合,符号誤り率とSN比の関係からPSK方式がASKやFSKと比較して有利とされる根拠を説明せよ.(H28,H23,H21)}
 ASK方式はリミッタをかけられないため,SN比が3dB低下する.\\
 FSK方式はリミッタをかけることはできるが,ASK方式の2倍の帯域を必要とするため,結果的にSN比はASKと同じである.\\
 PSK方式はリミッタをかけることができ,帯域はASK方式とおなじであるため,振幅性ノイズの改善が見込める.\\
 以上のことより,PSK方式が最も耐ノイズ性に優れているため,ASKやPSKと比較して有利とされる.\\

\subsection{スペクトラム拡散方式の特徴を2項目以上述べよ.(H30,H28)}
 広帯域に渡って低いSN比で通信するため,低電力である.\\
 ノイズに近い暗号化方式を用いて通信するため,セキュリティ面でも優れている.\\

\subsection{スペクトラム拡散方式の通信では擬似ランダム符号であるM系列が使用されるが,この符号を用いることによりどのような特徴が得られるか説明せよ.(H23)}
 M系列を用いると,信号が拡散され,低電力で信号を送信できるようになる.これによってSN比がノイズに埋もれるほどに小さくなるため高い秘匿性が得られる.また,複数のユーザーがいても互いに干渉することがないため,同じ帯域で同時に複数ユーザーが通信できる. \\

\newpage
\section{計算問題}
\subsection{図の$R$[$\mathrm{\Omega}$/m],$L$[H/m],$G$[S/m],$C$[F/m]の定数を持つ分布定数回路において,以下の電信方程式これをもとに正弦波に限定した波動方程式を求めよ.(H25,H24,H23)}
\vspace{7cm}

\subsection{分布定数回路における波動方程式の電圧解は次式で与えられる.この式を用い無限長線路での進行波速度である位相速度を求めよ.(H30,H28,H27,H26,H25,H24)}
\vspace{7cm}

\subsection{比誘電率 2 の絶縁物を使用した,内部導体の直径0.7mm,外部導体の内径3mmの同軸ケーブルがある.この同軸ケーブルの特性インピーダンスを求めよ.(H30,H28,H23)}
\vspace{7cm}

\newpage
\subsection{特性インピーダンス65$\mathrm{\Omega}$の平行線路に特性インピーダンス35$\mathrm{\Omega}$の線路を接続した.この時,接点における電圧反射係数$r$の値と35$\mathrm{\Omega}$側の電圧定在波比を求めよ.(H30)}
\vspace{7cm}

\subsection{TE$_10$モードの内径寸法比2:1,遮断周波数10GHzの方形導波管の内側寸法を求めよ.(H23)}
\vspace{7cm}

\subsection{TE$_10$モードの内径寸法が12.5mm*25mmの方形導波管の遮断周波数を求めよ.また,この導波管に10GHzの電磁波を通した場合の館内波長を求めよ.(H30,H28)}
\vspace{7cm}

\newpage
\subsection{地上から垂直に高さ80mのAM波用ユニポールアンテナを建てた.(H30)}
\subsubsection{このアンテナでもっとも効率よく電波を放射できる搬送周波数を求めよ.(H30)}
\vspace{7cm}
\subsubsection{このアンテナに高周波電流2.7Aを供給した時,水平方向の距離90kmの地点での電界強度と磁界強度を求めよ.(H30)}
\vspace{7cm}

\subsection{半波長ダイポールアンテナにおいて,1.5Aの高周波電流を給電した.正面水平方向の80km離れた地点の電界強度と磁界強度を求めよ.ただし,空間の電波伝搬固有インピーダンスを120$\pi\mathrm{\Omega}$とする.(H28)}
\vspace{7cm}

\newpage
\subsection{「1011 0111 0011」 と 「0011 0101 1011」 の符号間距離を求めよ (H30)}
\vspace{7cm}

\subsection{スペクトラム拡散方式を用いて受信側のSN比4倍,周波数帯域3MHzで通信した時の情報通信容量を求めよ.(H30,H28)}
\vspace{7cm}

\subsection{24kHzの搬送波1波長について位相変化12通り,振幅変動3通りの多値変調を行った.この場合のデータの情報容量を求めよ.(H23)}
\vspace{7cm}

\end{document}
